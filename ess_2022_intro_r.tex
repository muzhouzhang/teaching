\documentclass{beamer}
\usepackage[T1]{fontenc}
\usepackage[english]{babel}
\usefonttheme{serif}
\setbeamertemplate{navigation symbols}{
\usebeamerfont{footline}
\usebeamercolor[fg]{footline}
\insertframenumber/\inserttotalframenumber{}
}
\setbeamerfont{frametitle}{size = \small}
\usepackage{mathpazo}
\usepackage{float}
\usepackage[labelsep = colon]{caption}
\usepackage{amsmath}
\usepackage{setspace}
\usepackage{graphicx}
\usepackage{threeparttablex}
\usepackage{longtable}
\usepackage{booktabs}
\usepackage{dcolumn}
\usepackage{pdfpages}
\usepackage{ulem}

\title{Introduction to R}
\subtitle{Essex Summer School 2X}
\author{Muzhou Zhang}
\date{24 July 2022}

\begin{document}
\maketitle
\setstretch{1.25}

\begin{frame}{Contents}
\begin{itemize}
    \item basics: 1100--1215
    \item data management: 1230--1515
    \item descriptive \& statistical analysis: 1530--1700
    \item Q\&A at the end of each session if time permits
\end{itemize}
\end{frame}

\begin{frame}{Basics}
\framesubtitle{R}
R is a programming language for statistical computing
\begin{itemize}
    \item commands, no pull-down menu
    \item has its own strengths and limitations
    \item package-dependent\\ flexibility\\ reliability and reproducibility
\end{itemize}
\end{frame}

\begin{frame}{Basics}
\framesubtitle{RStudio}
RStudio is an IDE for R
\begin{itemize}
    \item limited GUI
    \item panes
    \item numerous keyboard shortcuts
    \item support for various script files
\end{itemize}
\end{frame}

\begin{frame}{Basics}
\framesubtitle{Today's course}
the tidyverse
\begin{itemize}
    \item popular in data science
    \item readable and intuitive syntax
    \item leapfrogging
\end{itemize}
\end{frame}    

\begin{frame}{Basics}
\framesubtitle{Getting started with R}
\begin{itemize}
    \item objects and ``types''
    \item assign: \texttt{<-}
    \item numeric, character, logical
    \item more than one type of ``types''; objects have classes too
\end{itemize}
\end{frame}

\begin{frame}{Basics}
\framesubtitle{Getting started with R}
\begin{itemize}
    \item functions and arguments
    \item type conversion
    \item concatenation, fundamental for automation: \texttt{c()}
    \item vectorization\\ with vectorization, how do we multiply matrices in R?
    \item packages: CRAN (stable release) and GitHub (development version)
    \item pipes: \texttt{\%>\%} (the tidyverse) or \texttt{|>} (base R)
\end{itemize}
\end{frame}    

\begin{frame}{Data management}
\framesubtitle{Import}
\begin{itemize}
    \item use \texttt{load()} to import \texttt{.RData}\\ do we \texttt{save()} regularly?
    \item use \texttt{haven} to import data in Stata, SPSS, or SAS format
    \item \texttt{readr} (CSV etc.), \texttt{readxl} (MS Excel), and lots more
    \item the example data is downloadable at: \texttt{https://tinyurl.com/ycu95adr}
    \item working directory: \texttt{setwd()}\\ absolute path: \texttt{/Users/mz/Desktop/GitHub/teaching/intro\_r/ess\_intro\_r\_maliniak\_2013.dta}\\ relative path: \texttt{ess\_intro\_r\_maliniak\_2013.dta}
\end{itemize}
\end{frame}

\begin{frame}{Data management}
\framesubtitle{Style}
\begin{itemize}
    \item use descriptive and short names for objects files
    \item use\_snake\_case (there are also CamelCase and kebab-case)
    \item use space as in regular English (except for the parentheses of function calls)
    \item use sensible line breaks and indents
    \item make code self-explanatory and comment important decisions
\end{itemize}
\end{frame}

\begin{frame}{Data management}
\framesubtitle{Example data: gender citation bias in IR}
Maliniak, D., Powers, R., \& Walter, B. (2013). The Gender Citation Gap in International Relations. International Organization, 67(4): 889--922
\begin{itemize}
    \item data file modified for pedagogic purpose
    \item outcome variable: number of citations of an article
    \item explanatory variable: gender composition of author(s)\\ all female, mixed-gender, all male
    \item now you have five minutes to have a look of the data
\end{itemize}
\end{frame}

\begin{frame}{Data management}
\framesubtitle{What's next}
\begin{itemize}
    \item \texttt{rename()}, \texttt{mutate()}, \texttt{filter()}, \texttt{drop\_na()}, \texttt{select()}
    \item logical operators: \texttt{\&} (AND), \texttt{|} (OR), \texttt{!} (NOT)
    \item conditional statements: \texttt{if\_else()} \sout{and the generalized \texttt{case\_when()}}
    \item \texttt{c()} and \texttt{\%in\%}
\end{itemize}
\end{frame}

\begin{frame}{Data management}
\framesubtitle{Rename \& create (replace) variables}
\begin{itemize}
    \item rename the outcome variable ``n\_cite''
    \item create two continuous variables indicating article's age and its squared term, respectively
    \item create a binary variable equals one if the first author from an American R1 university
    \item create a binary variable equals one if the journal one of the Top 3 (\textit{APSR, AJPS, JOP})
\end{itemize}
\end{frame}

\begin{frame}{Data management}
\framesubtitle{Drop observations \& keep variables}
\begin{itemize}
    \item drop articles on philosophy of sciences, political theory, and ``others''
    \item drop articles with a missing citation count
    \item keep selected variables only
    \item use pipes to connect everything
\end{itemize}
\end{frame}

\begin{frame}{Descriptive \& statistical analysis}
\framesubtitle{\texttt{ggplot2}}
\begin{itemize}
    \item start with \texttt{ggplot(the\_data)}
    \item add layer(s)
    \item add/adjust details
    \item we use \texttt{+} to connect different ``parts''
    \item I usually do not caption figures with \texttt{ggplot2}, why?
\end{itemize}
\end{frame}

\begin{frame}{Descriptive \& statistical analysis}
\framesubtitle{summary statistics}
\begin{itemize}
    \item no esay way of summarizing data and exporting the table ideally
    \item many packages are \LaTeX-centric
    \item class conversion
\end{itemize}
\end{frame}

\begin{frame}{Descriptive \& statistical analysis}
\framesubtitle{regression}
\begin{itemize}
    \item linear model (\texttt{lm}) and generalized linear models (\texttt{glm})
    \item panel data analysis (\texttt{plm}), multilevel models (\texttt{lme4}), negative binomial and ordered logistic/probit regression (\texttt{MASS}), design-based regression (\texttt{estimatr})
    \item presentation of results: table, coef plot, substantive effects
\end{itemize}
\end{frame}
    
\end{document}
